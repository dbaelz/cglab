% This document is free information. It comes without any warranty, to
% the extent permitted by applicable law. You can redistribute it
% and/or modify it under the terms of the Do What The Fuck You Want
% To Public License, Version 2, as published by Sam Hocevar. See
% http://sam.zoy.org/wtfpl/COPYING for more details.

\documentclass[a4paper,12pt]{article}

\usepackage[ngerman]{babel}
\usepackage[utf8]{inputenc}
\usepackage[T1]{fontenc}

\begin{document}

\title{Labor 4}
\author{Daniel Bälz, Fabian Freimüller, Florian Limberger}
\date{\today}

\maketitle

\section{Aufgabe 1}

Der Göpelarm hat die Länge $l=4,5m$, an dessen Ende ein Stützrad mit dem Radius $r=1,2m$ angebracht ist. Die Winkelgeschwindigkeit, mit der sich der Göpelarm dreht, beträgt
$$\omega = \frac{2 \times \pi}{60s}$$

Der Umfang des Stützrades beträgt
$$U = 2 \times \pi \times r = 2 \times \pi \times 1,2m = \pi \times 2,4m \approx 7,53m$$

Bei einer $360^{\circ}$-Drehung des Rades wird genau der Umfang des Rades als Kreisbogen zurückgelegt. Der dabei überstrichene Winkel $\alpha$ kann mit der Formel
$$\alpha = \frac{180 \times b}{\pi \times r}$$
berechnet werden. Für die Länge des Göpelarms ergibt sich dann ein überstrichener Winkel von $95.93^{\circ}$.

Daraus folgt, dass das Stützrad fur jedes Grad, um das der Göpelarm gedreht wird, um $\frac{360^{\circ}}{\alpha}$ rotiert, im konkreten Fall also um $3.75^{\circ}$.

Wenn nun eine Winkelgeschwindigkeit $\omega$ und eine Zeit $t$ gegeben sind, und man annimmt, dass sowohl der Göpelarm als auch das Stützrad zum Zeitpunkt $t_{0} = 0$ um $0^{\circ}$ rotiert waren, dann kann man die Rotation $r_{G}$ des Göpelarms mit
$$r_{G} = \frac{\omega \times t \times 180}{\pi}$$
und die Rotation $r_{S}$ des Stützrades mit
$$r_{S} = \frac{r_{G} \times 360^{\circ}}{\alpha}$$
berechnen.

Nun sei $f: N^2 \rightarrow N^2$ eine Funktion mit
$$f(t, \omega) = (r_{G}, r_{S})$$
welches unsere gesuchte Funktion ist.

\end{document}

