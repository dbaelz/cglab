% This document is free information. It comes without any warranty, to
% the extent permitted by applicable law. You can redistribute it
% and/or modify it under the terms of the Do What The Fuck You Want
% To Public License, Version 2, as published by Sam Hocevar. See
% http://sam.zoy.org/wtfpl/COPYING for more details.

\documentclass[a4paper,12pt]{article}

\usepackage[ngerman]{babel}
\usepackage[utf8]{inputenc}
\usepackage[T1]{fontenc}

\begin{document}

\title{Labor 4}
\author{Daniel Bälz, Fabian Freimüller, Florian Limberger}
\date{\today}

\maketitle

\section{Aufgabe 4a}
\subsection{Funktion}

Der Göpelarm hat die Länge $l=4,5m$, an dessen Ende ein Stützrad mit dem Radius
$r=1,2m$ angebracht ist. Die Winkelgeschwindigkeit, mit der sich der Göpelarm
dreht, beträgt $\omega = \frac{2 \times \pi}{60s}$

Der Umfang des Stützrades beträgt
\begin{eqnarray}
U & = & 2 \times \pi \times r \nonumber \\
& = & 2 \times \pi \times 1,2m \nonumber \\
& = & \pi \times 2,4m \nonumber \\
& \approx & 7,53m
\end{eqnarray}

Bei einer $360^{\circ}$-Drehung des Rades wird genau der Umfang des Rades als
Kreisbogen zurückgelegt. Der dabei überstrichene Winkel $\alpha$ kann mit der
Formel
\begin{equation}
\alpha = \frac{180 \times U}{\pi \times l}
\end{equation}
berechnet werden. Für die Länge des Göpelarms ergibt sich dann ein
überstrichener Winkel von $95.93^{\circ}$.

Daraus folgt, dass das Stützrad fur jedes Grad, um das der Göpelarm gedreht
wird, um $\frac{360^{\circ}}{\alpha}$ rotiert, im konkreten Fall also um
$3.75^{\circ}$.

Wenn nun eine Winkelgeschwindigkeit $\omega$ gegeben ist und man annimmt, dass
sowohl der Göpelarm als auch das Stützrad zum Zeitpunkt $t_{0} = 0$ um
$0^{\circ}$ rotiert sind, dann kann man die Rotationen als Funktionen von
$r: N \rightarrow N$ betrachten, welche die vergangene Zeit auf einen Winkel
abbilden. Konkret sind die Funktionenfür den Göpelarm
\begin{equation}
r_{G}(t) = \frac{\omega \times t \times 180}{\pi}
\end{equation}
für das Stützrad
\begin{equation}
r_{S}(t) = \frac{r_{G} \times 360^{\circ}}{\alpha}
\end{equation}.

Nun sei $f: N^2 \rightarrow N^2$ eine Funktion mit
$$f(t) = (r_{G}(t), r_{S}(t))$$
welches unsere gesuchte Funktion ist.

\subsection{Matrizen}
Die Rotationsmatrix in z-Richtung ist
\begin{equation}
R_{z} = \left(\begin{array}{ c c c c }
cos \gamma & sin \gamma & 0 & 0 \\
-sin \gamma & cos \gamma & 0 & 0 \\
0 & 0 & 1 & 0 \\
0 & 0 & 0 & 1
\end{array} \right)
\end{equation}
für den Winkel $\gamma$ und in y-Richtung
\begin{equation}
R_{y} = \left(\begin{array}{ c c c c }
cos \beta & 0 & -sin \beta & 0 \\
0 & 1 & 0 & 0 \\
sin \beta & 0 & cos \beta & 0 \\
0 & 0 & 0 & 1
\end{array} \right)
\end{equation}
für den Winkel $\beta$. Die Translation um die Länge l in
z-Richtung wird durch den Vektor
\begin{equation}
t_{z} = \left(\begin{array}{ c c c c }
1 & 0 & 0 & 0 \\
0 & 1 & 0 & 0 \\
0 & 0 & 1 & l \\
0 & 0 & 0 & 1
\end{array}\right)
\end{equation}
erreicht.

Da sich das Rad im Kreis drehen soll, muss es als erstes um den Winkel des
Göpelarms rotiert werden. Dann können die Translation auf und die Rotation um
die z-Achse angewendet werden. Die führt zu folgender Gleichung:
\begin{eqnarray}
P' & = & P \times R_{y} \times T_{z} \times R_{z} \nonumber \\
& = & P \times \left(\begin{array}{ c c c c }
\mbox{cos} \beta & 0 & - \mbox{sin} \beta & 0 \\
0 & 1 & 0 & 0 \\
\mbox{sin} \beta & 0 & \mbox{cos} \beta & 0 \\
0 & 0 & 0 & 1
\end{array} \right) \times \left(\begin{array}{ c c c c }
1 & 0 & 0 & 0 \\
0 & 1 & 0 & 0 \\
0 & 0 & 1 & l \\
0 & 0 & 0 & 1
\end{array} \right) \times \left(\begin{array}{ c c c c }
\mbox{cos} \gamma & \mbox{sin} \gamma & 0 & 0 \\
- \mbox{sin} \gamma & \mbox{cos} \gamma & 0 & 0 \\
0 & 0 & 1 & 0 \\
0 & 0 & 0 & 1
\end{array} \right) \nonumber \\
& = & P \times \left(\begin{array}{ c c c c }
\mbox{cos} \beta & 0 & - \mbox{sin} \beta & - \mbox{sin} \beta \times l \\
0 & 1 & 0 & 0 \\
\mbox{sin} \beta & 0 & \mbox{cos} \beta & \mbox{cos} \beta \times l \\
0 & 0 & 0 & 1
\end{array} \right) \times \left(\begin{array}{ c c c c }
\mbox{cos} \gamma & \mbox{sin} \gamma & 0 & 0 \\
- \mbox{sin} \gamma & \mbox{cos} \gamma & 0 & 0 \\
0 & 0 & 1 & 0 \\
0 & 0 & 0 & 1
\end{array} \right) \nonumber \\
& = & P \times \left(\begin{array}{ c c c c }
\mbox{cos} \beta \times \mbox{cos} \gamma &
\mbox{cos} \beta \times \mbox{sin} \gamma &
- \mbox{sin} \beta & - \mbox{sin} \beta \times l \\
- \mbox{sin} \gamma & \mbox{cos} \gamma & 0 & 0 \\
\mbox{sin} \beta \times \mbox{cos} \gamma &
\mbox{sin} \beta \times \mbox{sin} \gamma &
\mbox{cos} \beta & \mbox{cos} \beta \times l \\
0 & 0 & 0 & 1
\end{array} \right)
\end{eqnarray}

\end{document}

